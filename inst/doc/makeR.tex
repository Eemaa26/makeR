\documentclass[a4paper]{report}
\usepackage{RJournal}
\usepackage[round]{natbib}
\bibliographystyle{abbrvnat}

%% load any required packages here

\begin{document}

%% do not edit, for illustration only
\fancyhf{}
\fancyhead[LO,RE]{\textsc{Contributed Article}}
\fancyhead[RO,LE]{\thepage}
\fancyfoot[L]{The R Journal Vol. X/Y, Month, Year}
\fancyfoot[R]{ISSN 2073-4859}

%% replace RJtemplate with your article
\begin{article}
%%%%%%%%%%%%%%%%%%%%%%%%%%%%%

\title{makeR: An R Package for Managing Document Versions}
\author{by Jason M. Bryer}

\maketitle

\abstract{
\pkg{makeR} is an R package to help manage R Sweave projects where multiple versions are created based upon a single source repository. For example, a monthly report where each version is identical, with perhaps the exception of easily definable parameters (e.g. date ranges for data extraction, title, etc.). This package is not meant to assist with package development or more complex data analysis projects. For those types of projects, consider \pkg{devtools} \citep{devtools} or \pkg{ProjectTemplate} \citep{projecttemplate}, respectively.
}




\section{Section title in sentence case}

This section may contain a figure such as Figure \ref{figure:onecolfig}.

\begin{figure}
\vspace*{.1in}
\framebox[\textwidth]{\hfill \raisebox{-.45in}{\rule{0in}{1in}}
                      A picture goes here \hfill}
\caption{\label{figure:onecolfig}
A normal figure only occupies one column.}
\end{figure}

\section{Example}

The \href{http://www.r-bloggers.com}{R-Bloggers} site provides a wealth of information about R aggregated from many different R bloggers. Like many continuously changing data sources, we wish to periodically report on recent activity on the R-Bloggers site. 

\begin{example}
p <- newProject(name='MyProject')

\end{example}

\section{The project file format}

The \file{PROJECT.xml} file defines the properties, versions, and builds of a particular project. The file created and edited using the \pkg{XML} \citep{xml} package. Although the \pkg{makeR} package provides functions to manage projects, using XML allows for other R programmers to interact with the \pkg{makeR} project framework. Figure \ref{figure:projectxml} represents the contents of \file{PROJECT.xml} after running \code{demo(makeR)}.

\begin{figure*}
\vspace*{.1in}
%\framebox[\textwidth]{\hfill \raisebox{-.45in}{\rule{0in}{1in}}
\begin{example} 
<?xml version="1.0"?>
<project name="RBloggers" buildDir="build" releaseDir="release" sourceDir="source">
  <property name="email" value="GOOGLE READER USERNAME" type="character"/>
  <property name="passwd" value="GOOGLE READER PASSWORD" type="character"/>
  <versions>
    <version name="2011-12" major="1" minor="1">
      <property name="startDate" value="2011-12-01" type="character"/>
      <property name="endDate" value="2011-12-31" type="character"/>
    </version>
  </versions>
  <builds>
    <build major="1" minor="0" build="1" name="2011-12" timestamp="Wed Jan 18 12:29:51 2012" 
        R="R version 2.14.0 (2011-10-31)" platform="x86_64-apple-darwin9.8.0" 
        nodename="Jason-Bryers-MacBook-Air-2.local" 
        user="jbryer" file="rbloggers.pdf"/>
  </builds>
</project>
\end{example}
%}
\caption{\label{figure:projectxml}
\file{PROJECT.xml} for the R-Bloggers demo project.}
\end{figure*}

\section{Summary}

This file is only a basic article template. For full details of \emph{The R Journal}
style and information on how to prepare your article for submission, see the
\href{http://journal.r-project.org/latex/RJauthorguide.pdf}{Instructions for Authors}.

\section{Package development}

\texttt{ProjectVersion} is hosted on \href{http://github.com/jbryer/ProjectVersion}{Github}. The latest development version can be installed using the \pkg{devtools} \citep{devtools} package:

\begin{example}
install_github('ProjectVersion', 'jbryer')
\end{example}

\bibliography{Bibliography}


%\begin{thebibliography}{1}
%\expandafter\ifx\csname natexlab\endcsname\relax\def\natexlab#1{#1}\fi
%\expandafter\ifx\csname url\endcsname\relax
 % \def\url#1{{\tt #1}}\fi

%\bibitem[Ihaka and Gentleman(1996)]{R:Ihaka+Gentleman:1996}
%R.~Ihaka and R.~Gentleman.
%\newblock R: A language for data analysis and graphics.
%\newblock {\em Journal of Computational and Graphical Statistics}, 5\penalty0
 % (3):\penalty0 299--314, 1996.
%\newblock URL \url{http://www.amstat.org/publications/jcgs/}.

%\end{thebibliography}

\address{Jason M. Bryer\\
  Excelsior College\\
  7 Columbia Circle\\
  Albany, NY 12203\\
  USA}\\
\email{jason@bryer.org}


%%%%%%%%%%%%%%%%%%%%%%%%%%%%%
\end{article}

\end{document}

